\documentclass[11pt,a4paper]{article}
\usepackage[ngerman]{babel}
\usepackage[TS1,T1]{fontenc}
\usepackage[utf8x]{inputenc}
\selectlanguage{ngerman}
\usepackage{theorem}
\usepackage[scaled=0.9]{helvet}
\usepackage{amsmath}
\usepackage{amssymb}
\usepackage{hyperref}
\usepackage{stmaryrd}
\usepackage{pgf,tikz}
\usepackage{relsize}
\usepackage{enumitem}
\usepackage{graphicx}
\usepackage{algpseudocode,amsmath,xifthen}

\newcounter{numb}
\theoremstyle{break}
\theorembodyfont{\upshape}
   	\newtheorem{exercise}{Exercise}[numb]
\setlength{\oddsidemargin}{0cm}
\setlength{\textwidth}{16cm}
\setlength{\textheight}{23cm}
\setlength{\topmargin}{-2cm}

\usetikzlibrary{shapes,arrows,automata,positioning,decorations.fractals}
\renewcommand\familydefault{\sfdefault}

\newcommand{\header}[2]{
\begin{minipage}[b]{\textwidth}
		\parbox[t]{5cm}{%
			\includegraphics[width=4cm]{unilogo}
			\hfill
		}
		\parbox[b]{11cm}{%
			%\scshape%
			Heinrich-Heine-University D\"usseldorf\\
			Computer Science Department\\
			Software Engineering and Programming Languages\\
			%Professor Dr.\ M.\ Leuschel
		Philipp K\"orner
}
		
\end{minipage}
	\begin{center}
		\bf
		Functional Programming -- WT 2024 / 2025\\
        Reading Guide {\thenumb}: {#1}
	\end{center}

    \pagestyle{empty}
    \paragraph{Timeline:} This unit should be completed by {#2}.
}

\setcounter{numb}{12}


\begin{document}

\begin{minipage}[b]{\textwidth}
	\parbox[t]{5cm}{%
		\includegraphics[width=4cm]{unilogo}
		\hfill
	}
	\parbox[b]{11cm}{%
		%\scshape%
		Heinrich-Heine-University D\"usseldorf\\
		Computer Science Department\\
		Software Engineering and Programming Languages\\
		%Professor Dr.\ M.\ Leuschel
		Philipp K\"orner
	}
\end{minipage}
\begin{center}
	\bf
	Functional Programming -- WT 2021 / 2022\\
	Reading Guide 12: Polymorphism
\end{center}

\pagestyle{empty}

\section{Material} 

\begin{itemize}
    \item Alternatives:
        \begin{itemize}
\item Clojure for the Brave and True, chapter 13
\item 06\_polymorphism.clj
        \end{itemize}
    \item Reference:
        \begin{itemize}
\item Clojure Reference: Multimethods \url{https://clojure.org/reference/multimethods}
\item Clojure Reference: Protocols \url{https://clojure.org/reference/protocols}
        \end{itemize}
\end{itemize}

\paragraph{Timeline:} This unit should be completed by 17.01.2021.


\section{Learning Outcomes}

After completing this unit you should be able to

\begin{itemize}
    \item explain how multimethods and protocols solve the expression problem.
    \item use multimethods and protocols.
    \item compare the expressiveness of multimethods and protocols.
    \item recall and compare the performance of multimethods and protocols.
\end{itemize}

\section{Highlights}

\begin{itemize}
    \item Multimethods
    \item Protocols
    \item Functions / Macros: \verb|defmulti|, \verb|defmethod|, \verb|defprotocol|, \verb|extend-protocol|, \verb|extend-type|, \verb|extend|
\end{itemize}



\section{Exercises}


\begin{exercise}[Marsrover]
	In this task you are asked to implement a simplified version of the Marsrover-kata.
	For this, you are to write the control software for a robot that has landed on Mars.
	The robot has already scanned a rectangular area for this purpose, which unfortunately is surrounded by obstacles so that the robot cannot leave it.
	The requirements are as follows:

\begin{itemize}
    \item A surface area is given, e.g.
        \begin{verbatim}
      [["x" "x" "x" "x"]
       ["x" " " " " "x"]
       ["x" " " " " "x"]
       ["x" "x" "x" "x"]]
\end{verbatim}

	The data structure is a sequence of sequences that specifies the map with obstacles.
  	The first sequence (or line) is the northernmost line of the surface, the leftmost entry corresponds to the westernmost coordinate.
  	Entries that are the string ''x'' are obstacles, strings with whitespace are free (accessible) fields.
    \item An initial position of the rover and its orientation (\verb|:north|, \verb|:south|, \verb|:east|, \verb|:west|) is given.
    \item The signature for initialization is thus: \verb|(init! x y orientation surface)|
    \item Implement commands to move the rover forwards (f) and backwards (b).
    \item Implement command to rotate the rover by 90 degrees left and right (l, r).
    \item The rover is controlled via a string of commands (e.g. ''flffr'').
    \item The state of the rover is manipulated by \verb|(execute! string-sequence)|.
    \item If there is an obstacle in the way, the rover should abort further execution of the command sequence and report the position of the obstacle.
    \item The function \verb|rover-status| should return the tuple \verb|[x y direction]| with current information about the rover.
    \item Your solution must allow the addition of further commands,
        \emph{without the need to modify existing code}.
\end{itemize}

        Note: Use multimethods.
\end{exercise}


\begin{exercise}[Operator Overloading]

    Define a new protocol \verb|Multipliable| with a single function \verb|mult| that takes two arguments.
    The function \verb|mult| shall dispatch on the type of the first argument
    and ``multiply'' it with the second arguments.
    Calls and results may, for example, look like the following:

\begin{verbatim}
user=> (mult 3 4)
12
user=> (mult "foo" 4)
"foofoofoofoo"
user=> (mult '(1 2 3) 4)
(1 2 3 1 2 3 1 2 3 1 2 3)
\end{verbatim}

    Implement the protocol for at least these three types.
\end{exercise}

%\begin{exercise}[Lazy Searching - 4clojure Nr. 108]
%
%Implement a function that receives any number of ascending sequences
%as input and returns the smallest element that occurs in all sequences.
%The inputs can be infinite sequences.
%
%\begin{verbatim}
%(common-min [3 4 5])
%=> 3
%(common-min [1 2 3 4 5 6 7] [0.5 3/2 4 19])
%=> 4
%(common-min (range) (range 0 100 7/6) [2 3 5 7 11 13])
%=> 7
%(common-min (map (fn [x] (* x x x)) (range)) ;; Cubic numbers
%            (filter (fn [x] (zero? (bit-and x (dec x)))) (range)) ;; Powers of two
%            (iterate inc 20)) ;; Numbers greater than/equal to 20
%=> 64
%\end{verbatim}
%
%\end{exercise}

	\section*{Questions}
	If you have any questions, please contact Philipp K"orner (\texttt{p.koerner@hhu.de}) or post it to the Rocket.Chat group.
\end{document}

