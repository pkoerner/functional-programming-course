\documentclass[11pt,a4paper]{article}
\usepackage[ngerman]{babel}
\usepackage[TS1,T1]{fontenc}
\usepackage[utf8x]{inputenc}
\selectlanguage{ngerman}
\usepackage{theorem}
\usepackage[scaled=0.9]{helvet}
\usepackage{amsmath}
\usepackage{amssymb}
\usepackage{hyperref}
\usepackage{stmaryrd}
\usepackage{pgf,tikz}
\usepackage{relsize}
\usepackage{enumitem}
\usepackage{graphicx}
\usepackage{algpseudocode,amsmath,xifthen}

\newcounter{numb}
\theoremstyle{break}
\theorembodyfont{\upshape}
   	\newtheorem{exercise}{Exercise}[numb]
\setlength{\oddsidemargin}{0cm}
\setlength{\textwidth}{16cm}
\setlength{\textheight}{23cm}
\setlength{\topmargin}{-2cm}

\usetikzlibrary{shapes,arrows,automata,positioning,decorations.fractals}
\renewcommand\familydefault{\sfdefault}

\newcommand{\header}[2]{
\begin{minipage}[b]{\textwidth}
		\parbox[t]{5cm}{%
			\includegraphics[width=4cm]{unilogo}
			\hfill
		}
		\parbox[b]{11cm}{%
			%\scshape%
			Heinrich-Heine-University D\"usseldorf\\
			Computer Science Department\\
			Software Engineering and Programming Languages\\
			%Professor Dr.\ M.\ Leuschel
		Philipp K\"orner
}
		
\end{minipage}
	\begin{center}
		\bf
		Functional Programming -- WT 2024 / 2025\\
        Reading Guide {\thenumb}: {#1}
	\end{center}

    \pagestyle{empty}
    \paragraph{Timeline:} This unit should be completed by {#2}.
}

\setcounter{numb}{3}


\begin{document}

\begin{minipage}[b]{\textwidth}
	\parbox[t]{5cm}{%
		\includegraphics[width=4cm]{unilogo}
		\hfill
	}
	\parbox[b]{11cm}{%
		%\scshape%
		Heinrich-Heine-University D\"usseldorf\\
		Computer Science Department\\
		Software Engineering and Programming Languages\\
		%Professor Dr.\ M.\ Leuschel
		Philipp K\"orner
	}
\end{minipage}
\begin{center}
	\bf
	Functional Programming -- WT 2021 / 2022\\
	Reading Guide 3: Higher-Order Functions
\end{center}

\pagestyle{empty}

\section{Material} 

\begin{itemize}
    \item Alternatives:
        \begin{itemize}
            \item Clojure for the Brave and True, chapter 4 (Seq Function Examples + Function Functions, more in-depth explanations)
            \item 23\_hof.clj (basic coverage)
        \end{itemize}
    \item self-driven exercises!
\end{itemize}

\paragraph{Timeline:} This unit should be completed by 01.11.2021.

\paragraph{Note:} This week gives you an overview of a very important concept.
Thus, the material is rather little.
This does not mean that there is less to do this week: you should focus on practical aspects instead.
The REPL session contains pointers on what exercises you should attempt to deepen your understanding.


\section{Learning Outcomes}

After completing this unit you should be able to

\begin{itemize}
	%\item determine the evaluation order of elements in a function call.
    %\item define functions with local bindings.
    %\item use the fundamental control structures in small functions.
    \item use built-in higher-order functions.
    \item write higher-order functions.
    %\item recall fundamental sequence operations provided by the standard library.
\end{itemize}

\section{Highlights}

\begin{itemize}
    %\item $\beta$-reduction as evaluation-mechanism
    %\item Lists and sequences
    \item Higher-Order functions: concept, \verb|map|, \verb|filter|, \verb|reduce|, \verb|apply|, \verb|partial|
    %\item Special forms: \verb|def|, \verb|fn|, \verb|if|, \verb|let|, \verb|do|
    %\item Sequence operations: \verb|first|, \verb|rest|, \verb|range|, \verb|cons|, \verb|conj|, \verb|assoc|, \verb|dissoc|, \verb|disj|
    %\item Macros: \verb|ns|, \verb|for|, \verb|cond|, \verb|and|, \verb|or|
\end{itemize}



\section{Exercises}
\begin{exercise}[4clojure Exercise Unlocks --- Recommended!]
    After completing this unit, you gained the knowledge to solve the following exercises:

    \begin{itemize}
        \item elementary: 17--18, 64
        \item easy: 19--25, 27, 29--33, 38, 42, 45
        \item medium: 46, 59 (highly recommended!)
    \end{itemize}

    Note that in some exercises you should re-implement a Clojure built-in without calling it or related functions.
    Please also note that it is not expected that you solve every single one of these problems right now.
    Next week will unlock only a single new problem.
\end{exercise}



	\section*{Questions}
	If you have any questions, please contact Philipp K"orner (\texttt{p.koerner@hhu.de}) or post it to the Rocket.Chat group.
\end{document}

