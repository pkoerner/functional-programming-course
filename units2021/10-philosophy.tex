\documentclass[11pt,a4paper]{article}
\usepackage[ngerman]{babel}
\usepackage[TS1,T1]{fontenc}
\usepackage[utf8x]{inputenc}
\selectlanguage{ngerman}
\usepackage{theorem}
\usepackage[scaled=0.9]{helvet}
\usepackage{amsmath}
\usepackage{amssymb}
\usepackage{hyperref}
\usepackage{stmaryrd}
\usepackage{pgf,tikz}
\usepackage{relsize}
\usepackage{enumitem}
\usepackage{graphicx}
\usepackage{algpseudocode,amsmath,xifthen}

\newcounter{numb}
\theoremstyle{break}
\theorembodyfont{\upshape}
   	\newtheorem{exercise}{Exercise}[numb]
\setlength{\oddsidemargin}{0cm}
\setlength{\textwidth}{16cm}
\setlength{\textheight}{23cm}
\setlength{\topmargin}{-2cm}

\usetikzlibrary{shapes,arrows,automata,positioning,decorations.fractals}
\renewcommand\familydefault{\sfdefault}

\newcommand{\header}[2]{
\begin{minipage}[b]{\textwidth}
		\parbox[t]{5cm}{%
			\includegraphics[width=4cm]{unilogo}
			\hfill
		}
		\parbox[b]{11cm}{%
			%\scshape%
			Heinrich-Heine-University D\"usseldorf\\
			Computer Science Department\\
			Software Engineering and Programming Languages\\
			%Professor Dr.\ M.\ Leuschel
		Philipp K\"orner
}
		
\end{minipage}
	\begin{center}
		\bf
		Functional Programming -- WT 2024 / 2025\\
        Reading Guide {\thenumb}: {#1}
	\end{center}

    \pagestyle{empty}
    \paragraph{Timeline:} This unit should be completed by {#2}.
}

\setcounter{numb}{10}


\begin{document}

\begin{minipage}[b]{\textwidth}
	\parbox[t]{5cm}{%
		\includegraphics[width=4cm]{unilogo}
		\hfill
	}
	\parbox[b]{11cm}{%
		%\scshape%
		Heinrich-Heine-University D\"usseldorf\\
		Computer Science Department\\
		Software Engineering and Programming Languages\\
		%Professor Dr.\ M.\ Leuschel
		Philipp K\"orner
	}
\end{minipage}
\begin{center}
	\bf
	Functional Programming -- WT 2021 / 2022\\
	Reading Guide 10: Philosophie
\end{center}

\pagestyle{empty}

\paragraph{Deadline:} This unit should be completed by 20.12.2021.

\section{Material} 

\begin{itemize}
\item Rich Hickey: Hammock-Driven Development \url{https://www.youtube.com/watch?v=f84n5oFoZBc}
\item Stuart Halloway: Debugging with the Scientific Method \url{https://www.youtube.com/watch?v=FihU5JxmnBg}
\item Rich Hickey: Effective Programs \url{https://www.youtube.com/watch?v=2V1FtfBDsLU}
\end{itemize}


\section{Learning Outcomes}

After completing this unit you should be able to

\begin{itemize}
    \item classify the cost of a bug at different stages of development.
    \item assess architectural design in terms of mental capacity.
    \item explicitly specify trade-offs in the development of a particular project.
    \item debug a program using the scientific method.
    \item assess for which problems functional languages and Clojure in particular are useful.
\end{itemize}



\section{Exercises}

\begin{exercise}[Functions]
\begin{enumerate}[label=\alph*)]
\item The Clojure function \texttt{interleave} works as follows:
\begin{verbatim}
user=> (interleave [1 3 5] [2 4 6])
(1 2 3 4 5 6)
user=> (interleave [1 4] [2 5] [3 6])
(1 2 3 4 5 6)
\end{verbatim}

Write a function \texttt{(defn rev-interleave [s n] ...)}, that acts as \texttt{interleave}'s inverse function. 
\begin{verbatim}
user=> (rev-interleave [1 2 3 4 5 6] 2)
((1 3 5) (2 4 6))
user=> (rev-interleave [1 2 3 4 5 6] 3)
((1 4) (2 5) (3 6))
\end{verbatim}

If an input \texttt{s} cannot be split into \texttt{n} sequences of equal length, your function may behave arbitrarily, but it must terminate.
\item The Clojure function \texttt{flatten} works as follows:
\begin{verbatim}
user=> (flatten '[[1 2] 3 (4 [5 6])])
(1 2 3 4 5 6)
user=> (flatten ["a" ["b"] "c"])
("a" "b" "c")
user=> (flatten [1 [[[[[[[[[[[[2]]]]]]]]]]]]])
(1 2)
\end{verbatim}
Implement your own version of \texttt{flatten}, that additionally satisfies the following test case:
\begin{verbatim}
user=> (my-flatten [:a #{:b}])
(:a :b)
\end{verbatim}

Note: The function \verb|coll?| is useful to determine whether something is a scalar value or a collection.


\end{enumerate}

\end{exercise}

\begin{exercise}[Collatz sequence]
For any natural number n, the so-called Collatz sequence can be calculated.
The next term $x_{i+1} = c(x_i)$ in the sequence is obtained from the previous one by applying the following function:
\begin{equation*}
   c(n) =
   \begin{cases}
     3n+1 & \text{if } n \text{ odd} \\
     n/2  & \text{if } n \text{ even} \\
   \end{cases}
\end{equation*}

Write a function \texttt{(defn collatz [n] ...)}, which returns a sequence [n, c(n), c(c(n)), ..., 4, 2, 1]. For example, a call should look like this:

\begin{verbatim}
(collatz 11)
=> (11 34 17 52 26 13 40 20 10 5 16 8 4 2 1)
\end{verbatim}
\end{exercise}

\begin{exercise}[Combinations]

Implement a function that returns a sequence of all combinations of pairs of its elements. Duplicates need not be considered.

\begin{verbatim}
(combinations [1 2 3])
=> #{(1 2) (1 3) (2 3)}
(combinations [1 2 3 4])
=> #{(1 2) (1 3) (1 4) (2 3) (2 4) (3 4)}

\end{verbatim}
\end{exercise}

\section*{Questions}
If you have any questions, please contact Philipp K"orner (\texttt{p.koerner@hhu.de}).
\end{document}

