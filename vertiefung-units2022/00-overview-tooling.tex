\documentclass[11pt,a4paper]{article}
\usepackage[ngerman]{babel}
\usepackage[TS1,T1]{fontenc}
\usepackage[utf8x]{inputenc}
\selectlanguage{ngerman}
\usepackage{theorem}
\usepackage[scaled=0.9]{helvet}
\usepackage{amsmath}
\usepackage{amssymb}
\usepackage{hyperref}
\usepackage{stmaryrd}
\usepackage{pgf,tikz}
\usepackage{relsize}
\usepackage{enumitem}
\usepackage{graphicx}
\usepackage{algpseudocode,amsmath,xifthen}

\newcounter{numb}
\theoremstyle{break}
\theorembodyfont{\upshape}
   	\newtheorem{exercise}{Exercise}[numb]
\setlength{\oddsidemargin}{0cm}
\setlength{\textwidth}{16cm}
\setlength{\textheight}{23cm}
\setlength{\topmargin}{-2cm}

\usetikzlibrary{shapes,arrows,automata,positioning,decorations.fractals}
\renewcommand\familydefault{\sfdefault}

\newcommand{\header}[2]{
\begin{minipage}[b]{\textwidth}
		\parbox[t]{5cm}{%
			\includegraphics[width=4cm]{unilogo}
			\hfill
		}
		\parbox[b]{11cm}{%
			%\scshape%
			Heinrich-Heine-University D\"usseldorf\\
			Computer Science Department\\
			Software Engineering and Programming Languages\\
			%Professor Dr.\ M.\ Leuschel
		Philipp K\"orner
}
		
\end{minipage}
	\begin{center}
		\bf
		Functional Programming -- WT 2024 / 2025\\
        Reading Guide {\thenumb}: {#1}
	\end{center}

    \pagestyle{empty}
    \paragraph{Timeline:} This unit should be completed by {#2}.
}

\setcounter{numb}{0}


\begin{document}
	
\begin{minipage}[b]{\textwidth}
		\parbox[t]{5cm}{%
			\includegraphics[width=4cm]{unilogo}
			\hfill
		}
		\parbox[b]{11cm}{%
			%\scshape%
			Heinrich-Heine-University D\"usseldorf\\
			Computer Science Department\\
			Software Engineering and Programming Languages\\
			%Professor Dr.\ M.\ Leuschel
        Philipp K\"orner \\
        Jens Bendisposto
}
		
\end{minipage}
	\begin{center}
		\bf
		Functional Programming -- ST 2022\\
		Reading Guide 0: Overview and Tooling
	\end{center}
	
	\pagestyle{empty}
	
	\section*{Note}
	The learning objectives indicate what is expected of you in the exam.
	In principle it does not matter which material you consult to achieve the learning objective.
	The deadline allows you to meaningfully participate in the exercises.
	
	Solving the exercises is voluntary.
	The exercises in this guide present tools and as such are elementary for subsequent course material.
	In addition these tools are an immense help for solving the exercises.
	
	You are free to proceed to the next unit, as soon as you complete this one.  
	
	\paragraph{Timeline:} This unit should be completed by 12.04.2022.
	
	\section{Material} 
	
	\begin{itemize}
        \item The Clojure Toolbox \url{https://www.clojure-toolbox.com/}
        \item Deps and CLI Reference \url{https://www.clojure.org/reference/deps_and_cli}
        \item REPL: Material/intro.zip
	\end{itemize}
	
	
	\section{Learning Outcomes}
	
	After completing this unit you should be able to
	
	\begin{itemize}
        \item describe what the presented libraries are useful for.
        \item use the Clojure CLI tools.
	\end{itemize}
	
	\section{Exercises}

	\begin{exercise}[Tool Setup]
		Install the Clojure CLI Tools.
		Start a REPL-session using \verb|clj| as well. Integrate it in your editor.
	\end{exercise}
	
	
	
	
    \enlargethispage{2\baselineskip}
	\section*{Questions}
	If you have any questions, please contact Philipp K"orner (\texttt{p.koerner@hhu.de}) or post it to the Rocket.Chat group.
\end{document}

