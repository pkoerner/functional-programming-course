\documentclass[11pt,a4paper]{article}
\usepackage[ngerman]{babel}
\usepackage[TS1,T1]{fontenc}
\usepackage[utf8x]{inputenc}
\selectlanguage{ngerman}
\usepackage{theorem}
\usepackage[scaled=0.9]{helvet}
\usepackage{amsmath}
\usepackage{amssymb}
\usepackage{hyperref}
\usepackage{stmaryrd}
\usepackage{pgf,tikz}
\usepackage{relsize}
\usepackage{enumitem}
\usepackage{graphicx}
\usepackage{algpseudocode,amsmath,xifthen}

\newcounter{numb}
\theoremstyle{break}
\theorembodyfont{\upshape}
   	\newtheorem{exercise}{Exercise}[numb]
\setlength{\oddsidemargin}{0cm}
\setlength{\textwidth}{16cm}
\setlength{\textheight}{23cm}
\setlength{\topmargin}{-2cm}

\usetikzlibrary{shapes,arrows,automata,positioning,decorations.fractals}
\renewcommand\familydefault{\sfdefault}

\newcommand{\header}[2]{
\begin{minipage}[b]{\textwidth}
		\parbox[t]{5cm}{%
			\includegraphics[width=4cm]{unilogo}
			\hfill
		}
		\parbox[b]{11cm}{%
			%\scshape%
			Heinrich-Heine-University D\"usseldorf\\
			Computer Science Department\\
			Software Engineering and Programming Languages\\
			%Professor Dr.\ M.\ Leuschel
		Philipp K\"orner
}
		
\end{minipage}
	\begin{center}
		\bf
		Functional Programming -- WT 2024 / 2025\\
        Reading Guide {\thenumb}: {#1}
	\end{center}

    \pagestyle{empty}
    \paragraph{Timeline:} This unit should be completed by {#2}.
}

\setcounter{numb}{5}
\usepackage{csquotes}

\begin{document}

\begin{minipage}[b]{\textwidth}
	\parbox[t]{5cm}{%
		\includegraphics[width=4cm]{unilogo}
		\hfill
	}
	\parbox[b]{11cm}{%
		%\scshape%
		Heinrich-Heine-University D\"usseldorf\\
		Computer Science Department\\
		Software Engineering and Programming Languages\\
		%Professor Dr.\ M.\ Leuschel
		Philipp K\"orner \\
        Jens Bendisposto
	}
\end{minipage}
\begin{center}
	\bf
	Functional Programming -- ST 2022\\
    Reading Guide 06: Intro to Clojurescript
\end{center}

\pagestyle{empty}

\paragraph{Timeline:} This unit should be completed by 24.05.2022.

\section{Material} 

\renewcommand{\labelenumi}{\Alph{enumi}}
\begin{itemize}    
    \item Bruce Hauman - Developing ClojureScript With Figwheel \url{https://www.youtube.com/watch?v=j-kj2qwJa_E} (first 30 minutes)
        or, alternatively,  the Figwheel tutorial (\url{https://figwheel.org/tutorial}

    \item Shaun Mahood: re-frame your ClojureScript applications \url{https://www.youtube.com/watch?v=cDzjlx6otCU}
        or, alternatively,  the re-frame documentation (at least THE BASICS and MENTAL MODEL OMNIBUS of \url{https://day8.github.io/re-frame/re-frame/})
\end{itemize} 

\section{Learning Outcomes}

After completing this unit you should be able to

\begin{itemize}
    \item know how to interact with a ClojureScript REPL.
    \item use Figwheel in production.
    \item describe the basic state concept of re-frame.
\end{itemize}



\section{Exercises}

\begin{exercise}[Tooling]

    In the ILIAS, you will find Jan Ro\ss{}bachs unit on re-frame (06-reframe.zip).
    Work through the example/README.md file and ensure that all tools are installed.
    Then, you should be able to interact with the Clojurescript file (example/src/reframe/example/core.cljs).
    Try some change while the application runs and see whether it has an effect.


\end{exercise}


\end{document}

