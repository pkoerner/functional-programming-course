\usepackage[ngerman]{babel}
\usepackage[TS1,T1]{fontenc}
\usepackage[utf8x]{inputenc}
\selectlanguage{ngerman}
\usepackage{theorem}
\usepackage[scaled=0.9]{helvet}
\usepackage{amsmath}
\usepackage{amssymb}
\usepackage{hyperref}
\usepackage{stmaryrd}
\usepackage{pgf,tikz}
\usepackage{relsize}
\usepackage{enumitem}
\usepackage{graphicx}
\usepackage{algpseudocode,amsmath,xifthen}

\newcounter{numb}
\theoremstyle{break}
\theorembodyfont{\upshape}
   	\newtheorem{exercise}{Exercise}[numb]
\setlength{\oddsidemargin}{0cm}
\setlength{\textwidth}{16cm}
\setlength{\textheight}{23cm}
\setlength{\topmargin}{-2cm}

\usetikzlibrary{shapes,arrows,automata,positioning,decorations.fractals}
\renewcommand\familydefault{\sfdefault}
