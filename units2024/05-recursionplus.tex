\documentclass[11pt,a4paper]{article}
\usepackage[ngerman]{babel}
\usepackage[TS1,T1]{fontenc}
\usepackage[utf8x]{inputenc}
\selectlanguage{ngerman}
\usepackage{theorem}
\usepackage[scaled=0.9]{helvet}
\usepackage{amsmath}
\usepackage{amssymb}
\usepackage{hyperref}
\usepackage{stmaryrd}
\usepackage{pgf,tikz}
\usepackage{relsize}
\usepackage{enumitem}
\usepackage{graphicx}
\usepackage{algpseudocode,amsmath,xifthen}

\newcounter{numb}
\theoremstyle{break}
\theorembodyfont{\upshape}
   	\newtheorem{exercise}{Exercise}[numb]
\setlength{\oddsidemargin}{0cm}
\setlength{\textwidth}{16cm}
\setlength{\textheight}{23cm}
\setlength{\topmargin}{-2cm}

\usetikzlibrary{shapes,arrows,automata,positioning,decorations.fractals}
\renewcommand\familydefault{\sfdefault}

\newcommand{\header}[2]{
\begin{minipage}[b]{\textwidth}
		\parbox[t]{5cm}{%
			\includegraphics[width=4cm]{unilogo}
			\hfill
		}
		\parbox[b]{11cm}{%
			%\scshape%
			Heinrich-Heine-University D\"usseldorf\\
			Computer Science Department\\
			Software Engineering and Programming Languages\\
			%Professor Dr.\ M.\ Leuschel
		Philipp K\"orner
}
		
\end{minipage}
	\begin{center}
		\bf
		Functional Programming -- WT 2024 / 2025\\
        Reading Guide {\thenumb}: {#1}
	\end{center}

    \pagestyle{empty}
    \paragraph{Timeline:} This unit should be completed by {#2}.
}

\setcounter{numb}{5}


\begin{document}
	
\header{Recursion, Destructuring \& More}{11.11.2024}

\section{Material} 

\begin{itemize}
\item 05\_recursion.clj 
\item 04\_destructuring.clj
\item 07\_java\_interop.clj
\item 08\_namespaces.clj
\item 09\_evaluation\_order.clj
\item Learning Video: Destructuring: \url{https://mediathek.hhu.de/watch/b65ae856-dcd4-4eb7-82d7-62d0d727a525}
\item Learning Video: Recursion: \url{https://mediathek.hhu.de/watch/fe5c892a-b6e0-4811-aead-2821d5fa714e}
\end{itemize}

\paragraph{Note:} This unit groups several REPL sessions.
The focus lies on understanding explicit recursion and destructuring.
While the code volume is larger than usual,
the REPL sessions do not introduce complex concepts
and require less cognitive load.
In particular, we do not expect you to be able to write programs interacting heavily with Java
or to name all internal mappings of a namespace.
Nonetheless, you should be aware of the basic mechanisms
and might be required to work with them later on.

Please check the intended learning outcomes carefully before you take a deep dive into weird rabbit holes,
but --- of course! --- feel free to ask questions.

\section{Learning Outcomes}

After completing this unit you should be able to

\begin{itemize}
    \item write recursive programs yourself.
    \item read destructurings of data structures and specify the binding of symbols.
    \item be able to read Host interop forms.
    \item roughly sketch how namespaces work.
    \item know in which order symbols are evaluted.
\end{itemize}

\section{Highlights}

\begin{itemize}
    \item Destructuring
    \item Special Forms: \verb|loop|, \verb|recur|
    \item Functions: \verb|trampoline|
\end{itemize}



\section{Exercises}

\begin{exercise}[4clojure Exercise Unlocks]
    After completing this unit, you gained the knowledge to solve all exercises. Enjoy!

    Expect exam questions to be about ``medium'' level.
    It is worthwhile to tackle a couple hard problems 
    in order to prepare for the exam.
\end{exercise}

\begin{exercise}[Newton's method]
Newton's method is an algorithm to approximate a solution $c$ for $f(c)=0$ for a given function $f$.
The algorithm uses the series $$x_{n+1} = x_{n} - \frac{f(x_n)}{f'(x_n)}$$
It terminates if  $|x_{n+1} - x_n| < \epsilon$ holds for a given $\epsilon > 0$.

\begin{enumerate}[label=\alph*)]

  \item Write a function \texttt{(defn newton [f f'] ...)} that receives the function and its derivative as a parameter. The call to \texttt{newton} should then return a function which in turn receives a starting value $x$ and a precision $eps$ and approximates the solution for $f(x)=0 \pm eps$ using Newton's method.
  
   \item The square root of a number $K$ is a solution for the equation $c^2 - K = 0$. So we can apply Newton's Method to the function $f(x) = x^2 - K$. Write a function \texttt{(defn sqrt [n] ...)} which approximates the square root of $n$ using Newton's method. You can use $10^{-5}$ as both initial value and precision.
    
%  \item Newton's method is a fixed-point algorithm. Write a function \texttt{(defn fixedpoint [F guess eps?] ...)}, which calculates a fixed point starting from an initial value $guess$. The accuracy $eps?$ should itself be a function that receives two inputs: (the new and old value) and returns true if the two values match sufficiently well.

% \item Write Newton's method as instance of this fixed point function.

\emph{Spoiler alert}: we will return to this exercise in unit 13.
That reading guide will contain a solution.

\end{enumerate}
\end{exercise}

	\section*{Questions}
	If you have any questions, please contact Philipp K"orner (\texttt{p.koerner@hhu.de}) or post it to the Rocket.Chat group.
\end{document}

