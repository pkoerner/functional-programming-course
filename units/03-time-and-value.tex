\documentclass[11pt,a4paper]{article}
\usepackage[ngerman]{babel}
\usepackage[TS1,T1]{fontenc}
\usepackage[utf8x]{inputenc}
\selectlanguage{ngerman}
\usepackage{theorem}
\usepackage[scaled=0.9]{helvet}
\usepackage{amsmath}
\usepackage{amssymb}
\usepackage{hyperref}
\usepackage{stmaryrd}
\usepackage{pgf,tikz}
\usepackage{relsize}
\usepackage{enumitem}
\usepackage{graphicx}
\usepackage{algpseudocode,amsmath,xifthen}

\newcounter{numb}
\theoremstyle{break}
\theorembodyfont{\upshape}
   	\newtheorem{exercise}{Exercise}[numb]
\setlength{\oddsidemargin}{0cm}
\setlength{\textwidth}{16cm}
\setlength{\textheight}{23cm}
\setlength{\topmargin}{-2cm}

\usetikzlibrary{shapes,arrows,automata,positioning,decorations.fractals}
\renewcommand\familydefault{\sfdefault}

\newcommand{\header}[2]{
\begin{minipage}[b]{\textwidth}
		\parbox[t]{5cm}{%
			\includegraphics[width=4cm]{unilogo}
			\hfill
		}
		\parbox[b]{11cm}{%
			%\scshape%
			Heinrich-Heine-University D\"usseldorf\\
			Computer Science Department\\
			Software Engineering and Programming Languages\\
			%Professor Dr.\ M.\ Leuschel
		Philipp K\"orner
}
		
\end{minipage}
	\begin{center}
		\bf
		Functional Programming -- WT 2024 / 2025\\
        Reading Guide {\thenumb}: {#1}
	\end{center}

    \pagestyle{empty}
    \paragraph{Timeline:} This unit should be completed by {#2}.
}

\setcounter{numb}{3}


\begin{document}
	
\begin{minipage}[b]{\textwidth}
	\parbox[t]{5cm}{%
		\includegraphics[width=4cm]{unilogo}
		\hfill
	}
	\parbox[b]{11cm}{%
		%\scshape%
		Heinrich-Heine-University D\"usseldorf\\
		Computer Science Department\\
		Software Engineering and Programming Languages\\
		%Professor Dr.\ M.\ Leuschel
		Philipp K\"orner
	}
\end{minipage}
\begin{center}
	\bf
	Functional Programming -- WISE 2020 / 2021\\
	Reading Guide 3: Time and Value (Theory)
\end{center}

\pagestyle{empty}

\paragraph{Deadline:} This unit should be completed by 26.11.2020.

\section{Material} 

\begin{itemize}
\item Rich Hickey: Are We There Yet? \\ \url{https://www.infoq.com/presentations/Are-We-There-Yet-Rich-Hickey/}
\item Rich Hickey: Clojure Concurrency \url{https://youtu.be/nDAfZK8m5_8} (until 1:16:05)
\end{itemize}


\section{Learning Outcomes}

After completing this unit you should be able to

\begin{itemize}
	\item define and assign the terms identity, value observer and state.
    \item describe the epochal time-model and apply it to different implementations of state management.
    \item describe compare-and-swap (CAS)-semantics
    \item describe the role of persistent data structure in a concurrent context.
\end{itemize}


\section{Exercises}

\paragraph{Note:} Since this unit only deals with theoretical aspects, the exercises are used as an opportunity to practice and review older units.

\begin{exercise}[Newton's method]
Newton's method is an algorithm to approximate a solution $c$ for $f(c)=0$ for a given function $f$.
The algorithm uses the series $$x_{n+1} = x_{n} - \frac{f(x_n)}{f'(x_n)}$$
It terminates if  $|x_{n+1} - x_n| < \epsilon$ holds for a given $\epsilon > 0$.

\begin{enumerate}[label=\alph*)]

  \item Write a function \texttt{(defn newton [f f'] ...)} that receives the function and its derivative as a parameter. The call to \texttt{newton} should then return a function which in turn receives a starting value $x$ and a precision $eps$ and approximates the solution for $f(x)=0 \pm eps$ using Newton's method.
  
   \item The square root of a number $K$ is a solution for the equation $c^2 - K = 0$. So we can apply Newton's Method to the function $f(x) = x^2 - K$. Write a function \texttt{(defn sqrt [n] ...)} which approximates the square root of $n$ using Newton's method. You can use $10^{-5}$ as both initial value and precision.
    
%  \item Newton's method is a fixed-point algorithm. Write a function \texttt{(defn fixedpoint [F guess eps?] ...)}, which calculates a fixedpoint starting from an initial value $guess$. The accuracy $eps?$ should itself be a funciton that recieves two inputs: (the new and old value) and returns true if the two values match sufficiently well.

% \item Write Newton's method as instance of this fixed-point function.

\end{enumerate}
\end{exercise}


\begin{exercise}[Higher Order Functions]
\begin{enumerate}[label=\alph*)]

 \item Write a function \texttt{flip}, which reverses the arguments of a function.
 Examples:
 \begin{verbatim}
user=> (nth [3 4 5 6] 2)
5
user=> ((flip nth) 2 [3 4 5 6])
5
user=> (- 3 2 1)
0
user=> ((flip -) 1 2 3)
0
user=> (drop 2 [4 5 6 7])
(6 7)
user=> ((flip drop) [4 5 6 7] 2)
(6 7) \end{verbatim}
\item The Clojure function \texttt{comp} can be used to generate a composition of functions. The result of the call \texttt{((comp f g) x)} is the same as the the result of the call \texttt{(f (g x)))}. Implement your own version of \texttt{comp}. Your version should only cover the most general case \texttt{(defn mycomp [\& fs] ...)} instead of optimizing for various smaller arities like the Clojure version. 

\begin{verbatim}
user=> ((mycomp inc (fn [x] (* x x))) 4)
17
user=> ((mycomp (fn [x] (* x x)) inc) 4)
25 \end{verbatim}



\item Implement the Clojure function \texttt{juxt}, which receives functions as parameters and creates a new function. This function receives an input, applies each function to the input and creates a vector with the results. Implement the most general case of \texttt{juxt}.

\begin{verbatim}
user=> ((myjuxt inc dec (fn [x] (* x x))) 3)
[4 2 9] \end{verbatim}

\end{enumerate}
\end{exercise}



\begin{exercise}[\texttt{every?}]
    \texttt{(fn every? [f c])} is a higher-order function, which receives a function $f$ and a collection $c$ as parameters. 
    The return value is \texttt{true}, if the function $f$ returns a truthey value for every element in the collection and otherwise \texttt{false}.

    \begin{verbatim}
(every? even? [1 2 3 4]) 
=> false
(every? even? [2 4])
=> true \end{verbatim}

    Implement \texttt{every?} yourself.
\end{exercise}


\section*{Questions}
If you have any questions, please contact Philipp K"orner (\texttt{p.koerner@hhu.de}).
\end{document}

