\documentclass[11pt,a4paper]{article}

\usepackage[ngerman]{babel}
\usepackage[TS1,T1]{fontenc}
\usepackage[utf8x]{inputenc}
\usepackage{theorem}
\usepackage[scaled=0.9]{helvet}
\usepackage{amsmath}
\usepackage{amssymb}
\usepackage[T1]{fontenc}
\usepackage{hyperref}
\usepackage{stmaryrd}
\usepackage{pgf,tikz}
\usepackage{relsize}
\usepackage{enumitem}
\usepackage{graphicx}
\usepackage{upquote}
\usepackage{algpseudocode,amsmath,xifthen}

\newcounter{numb}
\theoremstyle{break}
\theorembodyfont{\upshape}
   	\newtheorem{aufgabe}{Aufgabe}[numb]
\setcounter{numb}{10}
\setlength{\oddsidemargin}{0cm}
\setlength{\textwidth}{16cm}
\setlength{\textheight}{23cm}
\setlength{\topmargin}{-2cm}

\usetikzlibrary{shapes,arrows,automata,positioning,decorations.fractals}
\renewcommand\familydefault{\sfdefault}


\begin{document}

\begin{minipage}[b]{\textwidth}
\parbox[t]{5cm}{%
\includegraphics[width=4cm]{unilogo}
\hfill
}
\parbox[b]{11cm}{%
%\scshape%
Heinrich-Heine-Universit\"at D\"usseldorf\\
Institut f\"ur Informatik\\
Lehrstuhl Softwaretechnik und Programmiersprachen\\
%Professor Dr.\ M.\ Leuschel
Philipp K\"orner
}

%%date
%\hfill 1.\@ August 2017\rule{0mm}{6mm}\quad\ %% <--
\end{minipage}
\begin{center}
\bf
Funktionale Programmierung -- WS 2020 / 2021\\
Reading Guide 10: clojure.spec
\end{center}

\pagestyle{empty}

\paragraph{Zeitliche Orientierung:} Diese Lerneinheit sollte bis zum 04.02.2021 abgeschlossen werden.
%\paragraph{Abgabe des Lerntagebuchs} \"uber das ILIAS bis zum 16.5.2020 mit unbegrenzt Material, Nachfrist bis zum 23.5.2020 mit zwei Seiten A4.

\section{Material} 

\begin{itemize}
    \item Getting Clojure, Kapitel 15
\item 14\_spec.clj
\item Clojure Guide: spec \url{https://clojure.org/guides/spec}
\item Rich Hickey: clojure.spec \url{https://vimeo.com/195711510}
\item Stuart Halloway: Agility \& Robustness: clojure.spec \url{https://www.youtube.com/watch?v=VNTQ-M_uSo8}
\end{itemize}


\section{Lernziele}

Nach dem Bearbeiten dieser Lerneinheit sollten Sie in der Lage sein

\begin{itemize}
    \item Vor- und Nachteile der clojure.spec Bibliothek zu diskutieren.
    \item Specs zu lesen und Beispielwerte anzugeben.
    \item selbst Specs zu schreiben.
\end{itemize}

\section{Highlights}

\begin{itemize}
    \item spec: \verb|def|, \verb|fdef|, \verb|valid?|,
    \item spec regex: \verb|cat|, \verb|alt|, \verb|+|, \verb|*|, \verb|?|, \verb|&|, \verb|spec|
    \item Specs kombinieren: \verb|and|, \verb|or|, \verb|map-of|, \verb|coll-of|, \verb|keys|
    \item conformed/unformed values
\end{itemize}



\section{Aufgaben}
\begin{aufgabe}[Specs]
\begin{enumerate}[label=\alph*)]
\item
Geben Sie eine Spec f\"ur einen Spieler in einem Kartenspiel an,
die in etwa wie folgt aussehen soll:

\begin{verbatim}
{:cards/name "Philipp"
 :cards/hand [[3 :clubs] [:ace :spades]]} 
\end{verbatim}

Als Name ist jeder beliebige String erlaubt.
Unter \verb|:cards/hand| ist eine nicht-leere Sequenz von 2-Tupeln assoziiert.
Der erste Eintrag ist der Wert der Karte, der zweite die Farbe.

\item
Schreiben Sie eine Funktion \verb|discard|,
die eine Reihe an Spielern,
sowie einen Namen und eine Spielkarte erh\"alt.
Ist diese Spielkarte tats\"achlich in der Hand
des genannten Spielers, wird diese daraus entfernt.

\item
Annotieren Sie \verb|discard| mit Specs,
die sicherstellen soll,
dass bei korrektem Aufruf wieder eine Reihe
an Spielern zur\"uckgegeben wird.

\item
Kann man mit generierten Tests ein Problem feststellen?

Anmerkung: man kann die Gr\"o\ss{}e der generierten Werte beispielsweise mit\\
\verb|(stest/check `discard {:clojure.spec.test.check/opts {:max-elements 4 :max-size 3}})|
limitieren. Ansonsten kann das Testing sehr lange dauern.
\end{enumerate}
\end{aufgabe}

\begin{aufgabe}[Specs II]
    In Aufgabe 9.2 wurden mit einem test.check Generator Benutzer mit Kennungen generiert.
    Schreiben Sie eine Spec f\"ur einen Benutzer.

    Ein Benutzer wird durch eine Map repr\"asentiert,
    die Vor- und Nachname sowie eine Kennung enth\"alt.
    Die Kennung besteht aus den ersten zwei Zeichen des Vornamens
    und den ersten drei Zeichen des Nachnamens und drei Ziffern.
    Ein Beispiel f\"ur einen validen Benutzer ist 
    
    \verb|{:vorname "John", :nachname "Wayne", :kennung "joway142"}|.

    Invalide Werte sind beispielsweise:
    \begin{itemize}
        \item \verb|42|
        \item \verb|{:vorname "John", :nachname "Wayne", :kennung "cowboystud69"} ; falsche Kennung|
        \item \verb|{:vorname "John", :last-name "Wayne", :kennung "joway142"} ; :last-name ist ungültig|
    \end{itemize}
\end{aufgabe}


\section*{Fragen}
Bei Fragen wenden Sie sich bitte an Philipp K"orner (\texttt{p.koerner@hhu.de}).
\end{document}

