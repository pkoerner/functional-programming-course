\documentclass[11pt,a4paper]{article}
\usepackage[ngerman]{babel}
\usepackage[TS1,T1]{fontenc}
\usepackage[utf8x]{inputenc}
\selectlanguage{ngerman}
\usepackage{theorem}
\usepackage[scaled=0.9]{helvet}
\usepackage{amsmath}
\usepackage{amssymb}
\usepackage{hyperref}
\usepackage{stmaryrd}
\usepackage{pgf,tikz}
\usepackage{relsize}
\usepackage{enumitem}
\usepackage{graphicx}
\usepackage{algpseudocode,amsmath,xifthen}

\newcounter{numb}
\theoremstyle{break}
\theorembodyfont{\upshape}
   	\newtheorem{exercise}{Exercise}[numb]
\setlength{\oddsidemargin}{0cm}
\setlength{\textwidth}{16cm}
\setlength{\textheight}{23cm}
\setlength{\topmargin}{-2cm}

\usetikzlibrary{shapes,arrows,automata,positioning,decorations.fractals}
\renewcommand\familydefault{\sfdefault}

\newcommand{\header}[2]{
\begin{minipage}[b]{\textwidth}
		\parbox[t]{5cm}{%
			\includegraphics[width=4cm]{unilogo}
			\hfill
		}
		\parbox[b]{11cm}{%
			%\scshape%
			Heinrich-Heine-University D\"usseldorf\\
			Computer Science Department\\
			Software Engineering and Programming Languages\\
			%Professor Dr.\ M.\ Leuschel
		Philipp K\"orner
}
		
\end{minipage}
	\begin{center}
		\bf
		Functional Programming -- WT 2024 / 2025\\
        Reading Guide {\thenumb}: {#1}
	\end{center}

    \pagestyle{empty}
    \paragraph{Timeline:} This unit should be completed by {#2}.
}

\setcounter{numb}{11}
\usepackage{mathtools}


\begin{document}

\begin{minipage}[b]{\textwidth}
	\parbox[t]{5cm}{%
		\includegraphics[width=4cm]{unilogo}
		\hfill
	}
	\parbox[b]{11cm}{%
		%\scshape%
		Heinrich-Heine-University D\"usseldorf\\
		Computer Science Department\\
		Software Engineering and Programming Languages\\
		%Professor Dr.\ M.\ Leuschel
		Philipp K\"orner
	}
\end{minipage}
\begin{center}
	\bf
	Functional Programming -- WISE 2020 / 2021\\
	Reading Guide 11: Transducer
\end{center}

\pagestyle{empty}

\paragraph{Deadline:} This unit should be completed by 11.02.2021.

\section{Material} 

\begin{itemize}
    \item 16\_transducer.clj
    \item Rich Hickey: Transducers \url{https://www.youtube.com/watch?v=6mTbuzafcII}
    \item Rich Hickey: Inside Transducers \url{https://www.youtube.com/watch?v=4KqUvG8HPYo}
\end{itemize}


\section{Learning Outcomes}

After completing this unit you should be able to

\begin{itemize}
    \item describe the idea behind transducers.
    \item understand and correctly use existing transducers.
    \item write basic (non-stateful) transducers yourself.
\end{itemize}

\section{Highlights}

\begin{itemize}
    \item reduce vs. transduce
    \item pass-through of step-functions
    \item state-flushing
\end{itemize}



\section{Exercises}

\begin{exercise}[Transducer]

Implement a function \verb|(transplace m)|,
which receives a map \verb|m| as argument and returns a transducer.
If an element is present as a key in \verb|m|,
it is replaced by the associated value,
otherwise the original element is retained.

The function \texttt{replace} must not be used for this.

\vspace{8px}
Example calls:
\begin{verbatim}
user=> (transduce (transplace {:y :a}) conj [:x :y :z])
[:x :a :z]
user=> (transduce (comp (transplace {nil 0}) (map inc))
                  conj
                  [42 nil 3])
[43 1 4]
user=> (transduce (comp (transplace {nil -1}) (partition-by pos?)) 
                  conj
                  [1 2 3 0 5 6])
[[1 2 3] [0] [5 6]]
\end{verbatim} 
\end{exercise} 

\begin{exercise}[Square-and-Multiply]

Powers of natural numbers can be calculated via $b^e = \underbrace{b * b * ... * b}_{e\text{ times}}$.
However, multiplications of two numbers are relatively slow on the CPU.

The idea behind exponentiation by squaring is to save multiplications by first calculating $b^{2^i}$ for $0 < 2^i \leq e$ and then multiplying the appropriate terms. These result from the binary representation of the exponent, for example $27 = 16 + 8 + 2 + 1$.

Consequently, $30^{27}$ is calculated using $30^{16} * 30^8 * 30^2 * 30^1 = 30^{2^4} * 30^{2^3} * 30^{2^1} * 30^{2^0}$.

Implement the algorithm. In the following section, a possible approach to one solution is described:

\begin{enumerate}[label=\alph*)]
\item
  Write a function \texttt{(defn to-bits [n] ...)} that takes a natural number and returns a list of 0s and 1s that is the binary representation of n.
\item 
    Write a function \texttt{(defn squares [b e] ...)} that, given a natural number b, computes the sequence \texttt{[$b^{2^0}$ $b^{2^1}$ $b^{2^2}$ ... $b^{2^j}$]} for $j \coloneqq \max\limits_{k} (2^k \leq e)$.
\item
  Combine the previous two functions suitably into one function \texttt{(defn square-and-multiply [b e] ...)}.
\end{enumerate}
\end{exercise}

\begin{exercise}[Hamiltonian path]
    A Hamiltonian path is a particular path through a graph:
    each vertex in the graph is visited exactly once.
    You are to implement a variant of a search for a Hamiltonian path.
    For simplicity, it is assumed that the starting vertex is known.

	The function \verb|hamilton| receives as its first argument
	a list of (directed) edges of the form \verb|[tail head]|,
	as its second argument the list of vertices in the graph,
	and as its third argument the vertex
	from which to start searching for a Hamiltonian path.
    For example, calls should look like this:

    \begin{verbatim}
    user=> (hamilton [[:a :b] [:b :c] [:c :d]]
                     #{:a :b :c} 
                     :a)
    (:a :b :c)
    user=> (hamilton [[:a :a] [:a :b] [:b :c] [:c :d]] 
                     #{:a :b :c}
                     :a) 
    (:a :b :c) ;; with loop [:a :a]
    user=> (hamilton (shuffle [[:a :a] [:a :d] [:b :c] [:a :b] [:c :d]])
                     #{:a :b :c :d} 
                     :a)
    (:a :b :c :d)
    user=> (hamilton [[:a :a] [:a :d] [:b :c] [:a :b] [:c :d]]
                     #{:a :b :c :d} 
                     :b)
    nil ;; no path
    \end{verbatim}

    Implement the function \verb|hamilton|.
    Do not naively enumerate all permutations of the vertices.
\end{exercise}

\section*{Questions}
If you have any questions, please contact Philipp K"orner (\texttt{p.koerner@hhu.de}).
\end{document}

