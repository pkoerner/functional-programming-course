\documentclass[11pt,a4paper]{article}

\usepackage[ngerman]{babel}
\usepackage[TS1,T1]{fontenc}
\usepackage[utf8x]{inputenc}
\usepackage{theorem}
\usepackage[scaled=0.9]{helvet}
\usepackage{amsmath}
\usepackage{amssymb}
\usepackage[T1]{fontenc}
\usepackage{hyperref}
\usepackage{stmaryrd}
\usepackage{pgf,tikz}
\usepackage{relsize}
\usepackage{enumitem}
\usepackage{graphicx}
\usepackage{algpseudocode,amsmath,xifthen}

\newcounter{numb}
\theoremstyle{break}
\theorembodyfont{\upshape}
   	\newtheorem{aufgabe}{Aufgabe}[numb]
\setcounter{numb}{4}
\setlength{\oddsidemargin}{0cm}
\setlength{\textwidth}{16cm}
\setlength{\textheight}{23cm}
\setlength{\topmargin}{-2cm}

\usetikzlibrary{shapes,arrows,automata,positioning,decorations.fractals}
\renewcommand\familydefault{\sfdefault}


\begin{document}

\begin{minipage}[b]{\textwidth}
\parbox[t]{5cm}{%
\includegraphics[width=4cm]{unilogo}
\hfill
}
\parbox[b]{11cm}{%
%\scshape%
Heinrich-Heine-Universit\"at D\"usseldorf\\
Institut f\"ur Informatik\\
Lehrstuhl Softwaretechnik und Programmiersprachen\\
%Professor Dr.\ M.\ Leuschel
Philipp K\"orner
}

%%date
%\hfill 1.\@ August 2017\rule{0mm}{6mm}\quad\ %% <--
\end{minipage}
\begin{center}
\bf
Funktionale Programmierung -- WS 2020 / 2021\\
Reading Guide 4: Zeit und Wert (Praxis)
\end{center}

\pagestyle{empty}

\paragraph{Zeitliche Orientierung:} Diese Lerneinheit sollte bis zum 03.12.2020 abgeschlossen werden.
%\paragraph{Abgabe des Lerntagebuchs} \"uber das ILIAS bis zum 16.5.2020 mit unbegrenzt Material, Nachfrist bis zum 23.5.2020 mit zwei Seiten A4.

\section{Material} 

\begin{itemize}
    \item 03\_concurrency.clj
    \item The Joy of Clojure, Kapitel 10.1 bis 10.4
    \item Clojure for the Brave and True, Kapitel 9
    \item Clojure for the Brave and True, Kapitel 10
    \item Clojure Reference: Atoms \url{https://clojure.org/reference/atoms}
    \item Clojure Reference: Refs and Transactions \url{https://clojure.org/reference/refs}
    \item Clojure Reference: Agents \url{https://clojure.org/reference/agents}
\end{itemize}


\section{Lernziele}

Nach dem Bearbeiten dieser Lerneinheit sollten Sie in der Lage sein

\begin{itemize}
    \item die Semantik von Konstrukten f\"ur Zeitmanagement zu beschreiben und zu vergleichen.
    \item das Konfliktverhalten von den Konstrukten f\"ur Zeitmanagement zu erl\"autern.
    \item fehlerhafte Verwendungen von Konstrukten f\"ur Zeitmanagement zu erkennen.
    \item jeweils Einsatzgebiete von Atomen, Refs und Agents zu identifizieren.
    \item jeweils Atome, Refs und Agenten konsistent zu verwenden.
    \item das Problem des write skews zu erkennen, zu beschreiben und zu l\"osen.
    \item die M\"achtigkeit (expressivity) von Atomen und Refs zu vergleichen.
    \item die Performance von Atomen und Refs zu kennen und zu vergleichen.
\end{itemize}

\section{Highlights}

\begin{itemize}
    \item Atoms, Refs, Agents
    \item Transaktionen
    \item write skew
    \item Macros: \verb|dosync|
    \item Funktionen: \verb|swap!|, \verb|reset!|, \verb|alter|, \verb|ref-set|, \verb|ensure|, \verb|deref|, \verb|send|, \verb|send-off|, \verb|await|
\end{itemize}



\section{Aufgaben}

\begin{aufgabe}[Minesweeper]
Implementieren Sie Minesweeper (siehe auch \url{https://de.wikipedia.org/wiki/Minesweeper}, \url{https://www.youtube.com/watch?v=LHY8NKj3RKs}). 

Mindestens folgende Funktionen sollen auf der REPL zur Verf\"ugung stehen:

\begin{itemize}
\item \texttt{(init! height width amount-of-mines)} soll das Spielfeld mit entsprechender H\"ohe, Breite und Minenanzahl initialisieren.
Weiterhin soll \texttt{(init!)} 30 als Standardwert f\"ur die H\"ohe, 16 f\"ur die Breite und 99 f\"ur die Minenanzahl verwenden.
\emph{Hinweis:} die Funktion \verb|shuffle| kann n\"utzlich sein.
\item \texttt{(reveal! x y)} deckt das Feld an der Koordinate (x,y) auf. Befindet sich dort eine Mine, ist das Spiel verloren.
Befindet sich keine Mine in der direkten Umgebung dieses Feld, sollen automatisch alle Felder darum herum aufgedeckt werden
(und analog f\"ur diese Felder diese Nachbarn, wenn es dort auch keine Minen in der N\"ahe gibt).
\item \texttt{(flag! x y)} soll das Feld (x,y) als Mine markieren oder diese Markierung entfernen, falls es bereits markiert ist.
Markierte Felder k\"onnen nicht mit \texttt{reveal!} aufgedeckt werden.
\item \texttt{(print-board!)} soll das Spielfeld in einem sinnvollen Format ausgeben.
\end{itemize}

\emph{Hinweis:} Sie ben\"otigen ein Konstrukt um Zustand zu managen. Achten Sie darauf, dass es zu keinen Inkonsistenzen kommt.

\end{aufgabe}


\section*{Fragen}
Bei Fragen wenden Sie sich bitte an Philipp K"orner (\texttt{p.koerner@hhu.de}).
\end{document}

