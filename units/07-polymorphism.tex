\documentclass[11pt,a4paper]{article}

\usepackage[ngerman]{babel}
\usepackage[TS1,T1]{fontenc}
\usepackage[utf8x]{inputenc}
\usepackage{theorem}
\usepackage[scaled=0.9]{helvet}
\usepackage{amsmath}
\usepackage{amssymb}
\usepackage[T1]{fontenc}
\usepackage{hyperref}
\usepackage{stmaryrd}
\usepackage{pgf,tikz}
\usepackage{relsize}
\usepackage{enumitem}
\usepackage{graphicx}
\usepackage{algpseudocode,amsmath,xifthen}

\newcounter{numb}
\theoremstyle{break}
\theorembodyfont{\upshape}
   	\newtheorem{aufgabe}{Aufgabe}[numb]
\setcounter{numb}{7}
\setlength{\oddsidemargin}{0cm}
\setlength{\textwidth}{16cm}
\setlength{\textheight}{23cm}
\setlength{\topmargin}{-2cm}

\usetikzlibrary{shapes,arrows,automata,positioning,decorations.fractals}
\renewcommand\familydefault{\sfdefault}


\begin{document}

\begin{minipage}[b]{\textwidth}
\parbox[t]{5cm}{%
\includegraphics[width=4cm]{unilogo}
\hfill
}
\parbox[b]{11cm}{%
%\scshape%
Heinrich-Heine-Universit\"at D\"usseldorf\\
Institut f\"ur Informatik\\
Lehrstuhl Softwaretechnik und Programmiersprachen\\
%Professor Dr.\ M.\ Leuschel
Philipp K\"orner
}

%%date
%\hfill 1.\@ August 2017\rule{0mm}{6mm}\quad\ %% <--
\end{minipage}
\begin{center}
\bf
Funktionale Programmierung -- WS 2020 / 2021\\
Reading Guide 7: Polymorphismus
\end{center}

\pagestyle{empty}

\paragraph{Zeitliche Orientierung:} Diese Lerneinheit sollte bis zum 14.01.2021 abgeschlossen werden.
%\paragraph{Abgabe des Lerntagebuchs} \"uber das ILIAS bis zum 16.5.2020 mit unbegrenzt Material, Nachfrist bis zum 23.5.2020 mit zwei Seiten A4.

\section{Material} 

\begin{itemize}
\item Clojure for the Brave and True, Kapitel 13
\item Clojure Reference: Multimethods \url{https://clojure.org/reference/multimethods}
\item Clojure Reference: Protocols \url{https://clojure.org/reference/protocols}
\item Philip Wadler: The Expression Problem \url{http://homepages.inf.ed.ac.uk/wadler/papers/expression/expression.txt}
\item 06\_concurrency.clj
\end{itemize}


\section{Lernziele}

Nach dem Bearbeiten dieser Lerneinheit sollten Sie in der Lage sein

\begin{itemize}
    \item das Expression Problem zu beschreiben.
    \item Einsatzgebiete sowie Vor- und Nachteile von Wrappern und Monkey Patching zu benennen.
    \item zu erkl\"aren, wie Multimethoden und Protokolle das Expression Problem l\"osen.
    \item Multimethoden und Protokolle zu verwenden.
    \item die M\"achtigkeit (expressivity) von Multimethoden und Protokollen zu vergleichen.
    \item die Performance zwischen Multimethoden und Protokollen zu kennen und zu vergleichen.
\end{itemize}

\section{Highlights}

\begin{itemize}
    \item Expression Problem
    \item Multimethod
    \item Protokolle
    \item Funktionen / Macros: \verb|defmulti|, \verb|defmethod|, \verb|defprotocol|, \verb|extend-protocol|, \verb|extend-type|, \verb|extend|
\end{itemize}



\section{Aufgaben}


\begin{aufgabe}[Marsrover]
In dieser Aufgabe soll eine vereinfachte Version des Marsrover-Katas implementiert werden.
Sie sollen daf\"ur die Steuerungssoftware f\"ur einen Roboter schreiben,
der auf dem Mars gelandet ist.
Der Roboter hat dazu schon einen rechteckigen Bereich abgescannt,
der leider von Hindernissen umgeben ist, sodass er diesen nicht verlassen kann.
Die Anforderungen lauten wie folgt:

\begin{itemize}
    \item Es wird eine Oberfl\"ache gegeben, z.B.
        \begin{verbatim}
      [["x" "x" "x" "x"]
       ["x" " " " " "x"]
       ["x" " " " " "x"]
       ["x" "x" "x" "x"]]
\end{verbatim}

        Die Datenstruktur ist eine Sequenz von Sequenzen, die die Karte mit Hindernissen vorgeben.
        Die erste enthaltene Sequenz (Zeile) ist die n\"ordlichste Linie der Oberfl\"ache, der
        linkeste Eintrag entspricht der westlichsten Koordinate.
        Eintr\"age, die der String "x" sind, sind Hindernisse, String mit einem Leerzeichen
        sind analog freie (befahrbare) Felder.
    \item Es wird eine initiale Position des Rovers und dessen Orientierung (\verb|:north|, \verb|:south|, \verb|:east|, \verb|:west|) gegeben.
    \item Die Signatur Initialisierung lautet also: \verb|(init! x y orientation surface)|
    \item Es sollen Befehle implementiert werden, um den Rover vorw\"arts (f) und r\"uckw\"arts (b) zu bewegen.
    \item Es sollen Befehle implementiert, um den Rover um 90 Grad nach links oder rechts zu drehen (l, r).
    \item Der Rover wird mit einem String gesteuert (z.B. "flffr").
    \item Es soll mit \verb|(execute! stringsequence)| der Zustand ver\"andert werden.
    \item Falls ein Hindernis im Weg liegt, soll der Rover die Befehlsequenz abbrechen
        und die Position des Hindernisses melden.
    \item Die Funktion \verb|rover-status| soll das Tupel \verb|[x y direction]| mit den derzeitigen Informationen \"uber den Rover zur\"uckgeben.
    \item Es muss m\"oglich sein, zu\"atzliche Befehle hinzuzuf\"ugen,
        \emph{ohne dass bestehender Code ver\"andert werden muss}.
\end{itemize}

        Hinweis: Verwenden Sie Multimethoden.
\end{aufgabe}


\begin{aufgabe}[Lazy Searching - 4clojure Nr. 108]

Implementieren Sie eine Funktion, die beliebig viele aufsteigend sortierte Sequenzen
als Eingabe bekommt und das kleinste Elemente zur\"uckgibt,
das in allen Sequenzen vorkommt.
Die Eingaben k\"onnen unendliche Sequenzen sein.

\begin{verbatim}
(common-min [3 4 5])
=> 3
(common-min [1 2 3 4 5 6 7] [0.5 3/2 4 19])
=> 4
(common-min (range) (range 0 100 7/6) [2 3 5 7 11 13])
=> 7
(common-min (map (fn [x] (* x x x)) (range)) ;; Kubikzahlen
            (filter (fn [x] (zero? (bit-and x (dec x)))) (range)) ;; Zweierpotenzen
            (iterate inc 20)) ;; Zahlen groesser/gleich 20
=> 64
\end{verbatim}

\end{aufgabe}

\section*{Fragen}
Bei Fragen wenden Sie sich bitte an Philipp K"orner (\texttt{p.koerner@hhu.de}).
\end{document}

