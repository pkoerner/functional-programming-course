\documentclass[11pt,a4paper]{article}
\usepackage[ngerman]{babel}
\usepackage[TS1,T1]{fontenc}
\usepackage[utf8x]{inputenc}
\selectlanguage{ngerman}
\usepackage{theorem}
\usepackage[scaled=0.9]{helvet}
\usepackage{amsmath}
\usepackage{amssymb}
\usepackage{hyperref}
\usepackage{stmaryrd}
\usepackage{pgf,tikz}
\usepackage{relsize}
\usepackage{enumitem}
\usepackage{graphicx}
\usepackage{algpseudocode,amsmath,xifthen}

\newcounter{numb}
\theoremstyle{break}
\theorembodyfont{\upshape}
   	\newtheorem{exercise}{Exercise}[numb]
\setlength{\oddsidemargin}{0cm}
\setlength{\textwidth}{16cm}
\setlength{\textheight}{23cm}
\setlength{\topmargin}{-2cm}

\usetikzlibrary{shapes,arrows,automata,positioning,decorations.fractals}
\renewcommand\familydefault{\sfdefault}

\newcommand{\header}[2]{
\begin{minipage}[b]{\textwidth}
		\parbox[t]{5cm}{%
			\includegraphics[width=4cm]{unilogo}
			\hfill
		}
		\parbox[b]{11cm}{%
			%\scshape%
			Heinrich-Heine-University D\"usseldorf\\
			Computer Science Department\\
			Software Engineering and Programming Languages\\
			%Professor Dr.\ M.\ Leuschel
		Philipp K\"orner
}
		
\end{minipage}
	\begin{center}
		\bf
		Functional Programming -- WT 2024 / 2025\\
        Reading Guide {\thenumb}: {#1}
	\end{center}

    \pagestyle{empty}
    \paragraph{Timeline:} This unit should be completed by {#2}.
}

\setcounter{numb}{12}


\begin{document}

\begin{minipage}[b]{\textwidth}
	\parbox[t]{5cm}{%
		\includegraphics[width=4cm]{unilogo}
		\hfill
	}
	\parbox[b]{11cm}{%
		%\scshape%
		Heinrich-Heine-University D\"usseldorf\\
		Computer Science Department\\
		Software Engineering and Programming Languages\\
		%Professor Dr.\ M.\ Leuschel
		Philipp K\"orner
	}
\end{minipage}
\begin{center}
	\bf
	Functional Programming -- WT 2023 / 2024\\
	Reading Guide 12: Time and Value (Praxis) --- Refs, Futures and Promises
\end{center}

\pagestyle{empty}

\paragraph{Timeline:} This unit should be completed by 15.01.2024.

\section{Material} 

\begin{itemize}
    \item Alternatives (note: not all books cover the write skew):
        \begin{itemize}
            \item 26\_refs.clj
            \item The Joy of Clojure, chapter 10.1 + 10.2
        \end{itemize}
            \item 27\_future\_promise.clj
    \item Clojure Reference: Refs and Transactions \url{https://clojure.org/reference/refs}
    \item Learning Video: Refs: \url{https://mediathek.hhu.de/watch/1134b0d5-32ea-4748-9c48-90c9f6bff394}
\end{itemize}


\section{Learning Outcomes}

After completing this unit you should be able to

\begin{itemize}
	\item describe and compare the semantics of time management constructs.
    \item explain the conflict behaviour of refs.
    \item identify incorrect use of refs.
    \item identify use cases for refs.
    \item use refs in a consistent manner, including commutative operations.
    \item identify, describe and solve the problem of write skew.
    \item compare the expressiveness of atoms and refs.
    \item remember and compare the performance of refs and atoms.
\end{itemize}

\section{Highlights}

\begin{itemize}
    \item refs 
    \item Transactions
    \item write skew
    \item Macros: \verb|dosync|
    \item Functions: \verb|alter|, \verb|ref-set|, \verb|ensure|, \verb|deref|
\end{itemize}



\section{Exercises}

No specific exercises this week. Feel free to tackle 4clojure problems or try your hand at the Advent of Code.

%\begin{exercise}[Minesweeper]
%Implement Minesweeper (see \url{https://de.wikipedia.org/wiki/Minesweeper}, \url{https://www.youtube.com/watch?v=LHY8NKj3RKs}). 
%
%At least the following functions shall be available on the REPL:
%
%\begin{itemize}
%\item \texttt{(init! height width amount-of-mines)} should initialize the playing field with corresponding height, width and number of mines.
%Additionally \texttt{(init!)} should use 30 as default value for the height, 16 for the width and 99 for the number of mines.
%\emph{Note:} the function \verb|shuffle| could be useful.
%\item \texttt{(reveal! x y)} uncovers the field at coordinate (x,y). If a mine happens to be there, the game is lost.
%If there is no mine in the surrounding area of this field, all fields around it should be uncovered automatically.
%(and analogously for the neighbours of those fields, if there are also no mines in the surrounding area).
%\item \texttt{(flag! x y)} should mark the field at (x,y) as mine or remove this mark if it already was marked.
%Marked fields cannot be uncovered by \texttt{reveal!}.
%\item \texttt{(print-board!)} should output the board in an appropriate format.
%\end{itemize}
%
%\emph{Note:} This requires the use of time management construct. Make sure that no inconsistent state can occur.
%
%\end{exercise}


	\section*{Questions}
	If you have any questions, please contact Philipp K"orner (\texttt{p.koerner@hhu.de}) or post it to the Rocket.Chat group.
\end{document}

