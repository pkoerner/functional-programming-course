\documentclass[11pt,a4paper]{article}
\usepackage[ngerman]{babel}
\usepackage[TS1,T1]{fontenc}
\usepackage[utf8x]{inputenc}
\selectlanguage{ngerman}
\usepackage{theorem}
\usepackage[scaled=0.9]{helvet}
\usepackage{amsmath}
\usepackage{amssymb}
\usepackage{hyperref}
\usepackage{stmaryrd}
\usepackage{pgf,tikz}
\usepackage{relsize}
\usepackage{enumitem}
\usepackage{graphicx}
\usepackage{algpseudocode,amsmath,xifthen}

\newcounter{numb}
\theoremstyle{break}
\theorembodyfont{\upshape}
   	\newtheorem{exercise}{Exercise}[numb]
\setlength{\oddsidemargin}{0cm}
\setlength{\textwidth}{16cm}
\setlength{\textheight}{23cm}
\setlength{\topmargin}{-2cm}

\usetikzlibrary{shapes,arrows,automata,positioning,decorations.fractals}
\renewcommand\familydefault{\sfdefault}

\newcommand{\header}[2]{
\begin{minipage}[b]{\textwidth}
		\parbox[t]{5cm}{%
			\includegraphics[width=4cm]{unilogo}
			\hfill
		}
		\parbox[b]{11cm}{%
			%\scshape%
			Heinrich-Heine-University D\"usseldorf\\
			Computer Science Department\\
			Software Engineering and Programming Languages\\
			%Professor Dr.\ M.\ Leuschel
		Philipp K\"orner
}
		
\end{minipage}
	\begin{center}
		\bf
		Functional Programming -- WT 2024 / 2025\\
        Reading Guide {\thenumb}: {#1}
	\end{center}

    \pagestyle{empty}
    \paragraph{Timeline:} This unit should be completed by {#2}.
}

\setcounter{numb}{2}

\begin{document}

\begin{minipage}[b]{\textwidth}
	\parbox[t]{5cm}{%
		\includegraphics[width=4cm]{unilogo}
		\hfill
	}
	\parbox[b]{11cm}{%
		%\scshape%
		Heinrich-Heine-University D\"usseldorf\\
		Computer Science Department\\
		Software Engineering and Programming Languages\\
		%Professor Dr.\ M.\ Leuschel
		Philipp K\"orner
	}
\end{minipage}
\begin{center}
	\bf
	Functional Programming -- WT 2023 / 2024\\
	Reading Guide 2: First Steps at Programming
\end{center}

\pagestyle{empty}

\section{Material} 

\begin{itemize}
    \item Learning Videos: 
        \begin{itemize}
            \item Lambda Calculus: \url{https://mediathek.hhu.de/watch/6cfddfc9-6d92-494f-90a0-8637b8b7f698}
            \item Syntax: \url{https://mediathek.hhu.de/watch/4aa60c78-23e5-48be-a5f6-b676dd7f2013}
            \item Coding Guidelines: \url{https://mediathek.hhu.de/watch/9b42cb61-e871-48c3-8ec8-0aca1f30a730}
        \end{itemize}
    \item Rich Hickey: Clojure for Java Programmers \url{https://www.youtube.com/watch?v=P76Vbsk_3J0} (until 1:35:38 --- introduction to the Clojure language and its motiviation --- optional revision and introduction)
    \item Alternative material:
        \begin{itemize}
            \item Clojure for the Brave and True, chapter 3 (covers all material)
            \item 01\_intro.clj (covers all material)
        \end{itemize}
    \item Please be aware of \url{https://clojure.org/api/cheatsheet} and that there are many useful functions.
        Especially the blocks ``Sequences'', ``Primitives'' and ``Collections'' are worth knowing.
        You do not have to learn everything by heart now, but go ahead and find out, what some functions do.
%\item Rich Hickey: Clojure for Java Programmers Part 2 \url{https://www.youtube.com/watch?v=hb3rurFxrZ8} (Overview over further topics of the lecture)
\end{itemize}

\paragraph{Timeline:} This unit should be completed by 23.10.2023.

\section{Learning Outcomes}

After completing this unit you should be able to

\begin{itemize}
	\item determine the evaluation order of elements in a function call.
    \item define functions with local bindings.
    \item use the fundamental control structures in small functions.
%    \item use built-in higher-order functions.
    %\item write higher-order functions.
    \item recall fundamental sequence operations provided by the standard library.
\end{itemize}

\section{Highlights}

\begin{itemize}
    \item $\beta$-reduction as evaluation-mechanism
    \item Lists and sequences
    %\item Higher-Order functions: concept, \verb|map|, \verb|filter|, \verb|reduce|, \verb|apply|
    \item Special forms: \verb|def|, \verb|fn|, \verb|if|, \verb|let|, \verb|do|
    \item Sequence operations: \verb|first|, \verb|rest|, \verb|range|, \verb|cons|, \verb|conj|, \verb|assoc|, \verb|dissoc|, \verb|disj|
    \item Macros: \verb|ns|, \verb|for|, \verb|cond|, \verb|and|, \verb|or|
\end{itemize}



\section{Exercises}

\paragraph{Notes}
Solving the exercises is voluntary.
Admission to the exam is solely based on passing the tests in the ILIAS.
Since programming is required in the exam
it is highly recommended to work on the exercises.

%In the ILIAS you can find templates for leiningen projects under Lerneinheiten > projektskelette.zip.
%Included there you will find functionality tests, which you can execute with \verb|lein test|\footnote{The Plug-in \url{https://github.com/weavejester/lein-auto} is useful in combination with this.}.
%
%You can initially define symbols with \verb|(declare seq-e fibonacci remove-duplicates)|
%to get rid of exceptions which are caused by the tests trying to reference undefined global variables and functions.
%Alternatively you can execute tests individually (see \verb|lein help test|).

\begin{exercise}[4clojure Exercise Unlocks --- Recommended!]
    After completing this unit, you gained the knowledge to solve the following exercises (\url{https://4clojure.oxal.org/} --- please bookmark for future reference):

    \begin{itemize}
        \item elementary: 4--13, 15--16, 35, 37, 57, 134, 145, 156, 161
    \end{itemize}
\end{exercise}

\begin{exercise}[Sequences]

In the following exercise simply define the described sequences. As Example: If you are to define the sequence of numbers between -100 and 100, a valid solution is: \texttt{(def example-seq (range -100 101))}
  

\begin{enumerate}[label=\alph*)]
  \item Define the sequence \verb|seq-a|, containing all integers from 100 to -100 in descending order.
  \item Define the sequence \verb|seq-b|, containing all square integers between 0 and 1000.
  \item Define the sequence \verb|seq-c|, containing all integers between 0 and 1000, which are not evenly divisible by 3.
  \item Define the sequence \verb|seq-d|, containing all tuples [n, m] of integers for which the following holds: $0 < n < 1000 \wedge n^2 < m \wedge \text{m is minimal}$.  The sequence begins with tuples [1,2], [2,5], [3,10].
  \item Define the sequence \verb|seq-e|, containing all integers with 5 digits (represented as 5 tuple), which are palindromes. Additionally the digits should be strictly ascending up to the middle digit. [0 2 3 2 0] is part of the sequence, but [0 3 2 3 0] is not, neither is [1 1 1 1 1].
  
    
  \end{enumerate}

    \paragraph{Hint} \verb|for| is a useful macro that is similar to list-comprehensions in Python. 

\end{exercise}





	\section*{Questions}
	If you have any questions, please contact Philipp K"orner (\texttt{p.koerner@hhu.de}) or post it to the Rocket.Chat group.
\end{document}

