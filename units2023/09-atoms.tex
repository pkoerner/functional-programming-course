\documentclass[11pt,a4paper]{article}
\usepackage[ngerman]{babel}
\usepackage[TS1,T1]{fontenc}
\usepackage[utf8x]{inputenc}
\selectlanguage{ngerman}
\usepackage{theorem}
\usepackage[scaled=0.9]{helvet}
\usepackage{amsmath}
\usepackage{amssymb}
\usepackage{hyperref}
\usepackage{stmaryrd}
\usepackage{pgf,tikz}
\usepackage{relsize}
\usepackage{enumitem}
\usepackage{graphicx}
\usepackage{algpseudocode,amsmath,xifthen}

\newcounter{numb}
\theoremstyle{break}
\theorembodyfont{\upshape}
   	\newtheorem{exercise}{Exercise}[numb]
\setlength{\oddsidemargin}{0cm}
\setlength{\textwidth}{16cm}
\setlength{\textheight}{23cm}
\setlength{\topmargin}{-2cm}

\usetikzlibrary{shapes,arrows,automata,positioning,decorations.fractals}
\renewcommand\familydefault{\sfdefault}

\newcommand{\header}[2]{
\begin{minipage}[b]{\textwidth}
		\parbox[t]{5cm}{%
			\includegraphics[width=4cm]{unilogo}
			\hfill
		}
		\parbox[b]{11cm}{%
			%\scshape%
			Heinrich-Heine-University D\"usseldorf\\
			Computer Science Department\\
			Software Engineering and Programming Languages\\
			%Professor Dr.\ M.\ Leuschel
		Philipp K\"orner
}
		
\end{minipage}
	\begin{center}
		\bf
		Functional Programming -- WT 2024 / 2025\\
        Reading Guide {\thenumb}: {#1}
	\end{center}

    \pagestyle{empty}
    \paragraph{Timeline:} This unit should be completed by {#2}.
}

\setcounter{numb}{9}


\begin{document}

\begin{minipage}[b]{\textwidth}
	\parbox[t]{5cm}{%
		\includegraphics[width=4cm]{unilogo}
		\hfill
	}
	\parbox[b]{11cm}{%
		%\scshape%
		Heinrich-Heine-University D\"usseldorf\\
		Computer Science Department\\
		Software Engineering and Programming Languages\\
		%Professor Dr.\ M.\ Leuschel
		Philipp K\"orner
	}
\end{minipage}
\begin{center}
	\bf
	Functional Programming -- WT 2023 / 2024\\
	Reading Guide 9: Time and Value (Praxis) --- Atoms and Agents
\end{center}

\pagestyle{empty}

\paragraph{Timeline:} This unit should be completed by 11.12.2023.

\section{Material} 

\begin{itemize}
    \item Alternatives:
        \begin{itemize}
            \item \begin{itemize}
                \item 24\_atoms.clj
                \item 25\_agents.clj
            \end{itemize}
            \item The Joy of Clojure, chapter 10.4 + 10.3
            \item Clojure for the Brave and True, chapter 10 (covers most of refs as well): Object-Oriented Metaphysics, Clojure Metaphysics, Atoms, Watches and Validators
        \end{itemize}
    \item Reference Pages:
        \begin{itemize}
    \item Clojure Reference: Atoms \url{https://clojure.org/reference/atoms}
    \item Clojure Reference: Agents \url{https://clojure.org/reference/agents}
        \end{itemize}
    \item Learning Videos:
        \begin{itemize}
    \item Atoms: \url{https://mediathek.hhu.de/watch/b97ca683-8682-41e1-a491-8e0168b09837}
    \item Agents: \url{https://mediathek.hhu.de/watch/45050c42-1c2d-41ed-a7f1-66d3bd181c08}
        \end{itemize}
\end{itemize}


\section{Learning Outcomes}

After completing this unit you should be able to

\begin{itemize}
	\item describe and compare the semantics of atoms and agents constructs.
    \item explain the conflict behaviour of atoms and agents constructs.
    \item identify incorrect usage of atoms and agents constructs.
    \item identify use cases for atoms and agents.
    \item use atoms and agents in a consistent manner.
\end{itemize}

\section{Highlights}

\begin{itemize}
    \item Atoms, agents
    \item Macros: \verb|dosync|
    \item Functions: \verb|swap!|, \verb|reset!|, \verb|deref|, \verb|send|, \verb|send-off|, \verb|await|
\end{itemize}



\section{Exercises}

\begin{exercise}[Minesweeper]
Implement Minesweeper (see \url{https://de.wikipedia.org/wiki/Minesweeper}, \url{https://www.youtube.com/watch?v=LHY8NKj3RKs}). 

At least the following functions shall be available on the REPL:

\begin{itemize}
\item \texttt{(init! height width amount-of-mines)} should initialize the playing field with corresponding height, width and number of mines.
Additionally \texttt{(init!)} should use 30 as default value for the height, 16 for the width and 99 for the number of mines.
\emph{Note:} the function \verb|shuffle| could be useful.
\item \texttt{(reveal! x y)} uncovers the field at coordinate (x,y). If a mine happens to be there, the game is lost.
If there is no mine in the surrounding area of this field, all fields around it should be uncovered automatically.
(and analogously for the neighbours of those fields, if there are also no mines in the surrounding area).
\item \texttt{(flag! x y)} should mark the field at (x,y) as mine or remove this mark if it already was marked.
Marked fields cannot be uncovered by \texttt{reveal!}.
\item \texttt{(print-board!)} should output the board in an appropriate format.
\end{itemize}

\emph{Note:} This requires the use of time management construct. Make sure that no inconsistent state can occur.

\emph{Note:} Yes, it is somewhat ugly to extract the fields surrounding a given field.

\end{exercise}


	\section*{Questions}
	If you have any questions, please contact Philipp K"orner (\texttt{p.koerner@hhu.de}) or post it to the Rocket.Chat group.
\end{document}

