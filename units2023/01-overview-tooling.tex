\documentclass[11pt,a4paper]{article}
\usepackage[ngerman]{babel}
\usepackage[TS1,T1]{fontenc}
\usepackage[utf8x]{inputenc}
\selectlanguage{ngerman}
\usepackage{theorem}
\usepackage[scaled=0.9]{helvet}
\usepackage{amsmath}
\usepackage{amssymb}
\usepackage{hyperref}
\usepackage{stmaryrd}
\usepackage{pgf,tikz}
\usepackage{relsize}
\usepackage{enumitem}
\usepackage{graphicx}
\usepackage{algpseudocode,amsmath,xifthen}

\newcounter{numb}
\theoremstyle{break}
\theorembodyfont{\upshape}
   	\newtheorem{exercise}{Exercise}[numb]
\setlength{\oddsidemargin}{0cm}
\setlength{\textwidth}{16cm}
\setlength{\textheight}{23cm}
\setlength{\topmargin}{-2cm}

\usetikzlibrary{shapes,arrows,automata,positioning,decorations.fractals}
\renewcommand\familydefault{\sfdefault}

\newcommand{\header}[2]{
\begin{minipage}[b]{\textwidth}
		\parbox[t]{5cm}{%
			\includegraphics[width=4cm]{unilogo}
			\hfill
		}
		\parbox[b]{11cm}{%
			%\scshape%
			Heinrich-Heine-University D\"usseldorf\\
			Computer Science Department\\
			Software Engineering and Programming Languages\\
			%Professor Dr.\ M.\ Leuschel
		Philipp K\"orner
}
		
\end{minipage}
	\begin{center}
		\bf
		Functional Programming -- WT 2024 / 2025\\
        Reading Guide {\thenumb}: {#1}
	\end{center}

    \pagestyle{empty}
    \paragraph{Timeline:} This unit should be completed by {#2}.
}

\setcounter{numb}{1}

\begin{document}
	
\begin{minipage}[b]{\textwidth}
		\parbox[t]{5cm}{%
			\includegraphics[width=4cm]{unilogo}
			\hfill
		}
		\parbox[b]{11cm}{%
			%\scshape%
			Heinrich-Heine-University D\"usseldorf\\
			Computer Science Department\\
			Software Engineering and Programming Languages\\
			%Professor Dr.\ M.\ Leuschel
		Philipp K\"orner
}
		
\end{minipage}
	\begin{center}
		\bf
		Functional Programming -- WT 2023 / 2024\\
		Reading Guide 1: Overview and Tooling
	\end{center}
	
	\pagestyle{empty}
	
	\section*{Note}
	The learning objectives indicate what is expected of you in the exam.
	In principle it does not matter which material you consult to achieve the learning objective.
	The deadline allows you to meaningfully participate in the exercises.
	
	Solving the exercises is voluntary.
	The exercises in this guide present tools and as such are elementary for subsequent course material.
	In addition these tools are an immense help for solving the exercises.
	
	You are free to proceed to the next unit, as soon as you complete this one.  
	
	\paragraph{Timeline:} This unit should be completed by 16.10.2023.
	
	\section{Material} 
	
	\begin{itemize}
        %(ca. 45 minutes (video edition))
        \item Leiningen (Clojure Project Management Tool --- install it) \url{https://leiningen.org/}
		%\item Clojure Cheatsheet \url{https://clojure.org/api/cheatsheet}
        \item REPL: 00-syntax.clj
        \item Optional:
            \begin{itemize}
                \item Clojure Newbie Guide (overview of Clojure-related resources --- give it a glance) \url{http://www.clojurenewbieguide.com/}
                \item The Joy of Clojure, Chapter 1 - Clojure Philosophy 
                    (Do not be discouraged if you do not understand everything from this chapter immediately.
                    The book is not especially suited for beginners, but gives a comprehensive overview of the topics we will focus on.)
                \item Janet A.\@ Carr --- Mindset Shifts for Functional Programming (with Clojure) \url{https://blog.janetacarr.com/mindset-shifts-for-functional-programming-with-clojure/}
                    (This short article remains relevant for the next few weeks and you will understand more once the course progresses. By the end of unit 6, you should understand the entire text.)
            \end{itemize}
	\end{itemize}
	
	
	\section{Learning Outcomes}
	
	After completing this unit you should be able to
	
	\begin{itemize}
		\item name the characteristics of functional programming.
		\item describe the downsides of object-oriented programming.
		\item use an editor with a REPL-plugin and paredit.
        \item read, write and understand fundamental Clojure syntax.
		%\item list some Clojure-functions and explain their functionality.
	\end{itemize}
	
	\section{Exercises}

	\begin{exercise}[Leiningen]
		Install Leiningen (\url{https://github.com/technomancy/leiningen}).
		Please use the installation instructions provided on the linked website
		to ensure you are not an outdated version of Clojure.
		
		Find out how to generate a new project
		and familiarise yourself with the directory structure.
		Start a REPL-session as well.
	\end{exercise}
	
	
	\begin{exercise}[Paredit]
		Paredit is an important aid in programming in a LISP.
		A paredit-plugin exists for most editors.
		Make sure that your editor of choice has such a plugin
		and install or activate it.
		
		
		Paredit helps you by keeping parentheses balanced.
		In order to work efficiently you should
		become acquainted with the keyboard shortcuts that allow for the following manipulations
		(copy \& paste or retyping are prohibited!):
		
		\begin{enumerate}[label=\alph*)]
			\item
			\texttt{(foo (bar) baz)} $\longrightarrow$ \texttt{(foo (bar baz))} \textit{(Right Slurp)}
			\item
			\texttt{(foo (bar baz))} $\longrightarrow$ \texttt{(foo bar (baz))} \textit{(Left Barf)}
			\item
			\texttt{(foo bar (baz))} $\longrightarrow$ \texttt{(foo (bar baz))} \textit{(Left Slurp)}
			\item
			\texttt{(foo (bar baz))} $\longrightarrow$ \texttt{(foo (bar) baz)} \textit{(Right Barf)}
			\item
			\texttt{(foo) (bar)} $\longrightarrow$ \texttt{(foo bar)} \textit{(Join)}
			\item
			\texttt{(foo bar)} $\longrightarrow$ \texttt{(foo) (bar)} \textit{(Split)}
			\item
			\texttt{foo} $\longrightarrow$ \texttt{(foo)} \textit{(Wrap)}
			\item
			\texttt{(foo (bar baz))} $\longrightarrow$ \texttt{(foo bar baz)} \textit{(Splice)}
			\item
			\texttt{(foo (bar baz))} $\longrightarrow$ \texttt{(foo)} \textit{(Delete / Killing)}
		\end{enumerate}
		
	\end{exercise}
	
	\begin{exercise}[REPL-Plugin]
		Many editors provide plugins
		to connect to open REPL sessions
		and execute code within their context.
		
		Find out if such a plugin exists for your chosen editor,
		start a REPL session and evaluate the following
		code (You do not have to understand the code at this point!)
		
		\begin{verbatim}
			(def my-seq
			    (lazy-cat [0 1] (map + (rest my-seq) my-seq)))
			
			(take 10 my-seq)
		\end{verbatim}
	\end{exercise}

    \begin{exercise}[4clojure Exercise Unlocks --- Recommended!]
        After completing this unit, you gained the knowledge to solve the following 4clojure exercises:

        \begin{itemize}
            \item elementary: 1--3, 162
        \end{itemize}

        \noindent You can find them in the ILIAS system in the ``Material/exercises.zip'' file or, alternatively,
        at \url{https://4clojure.oxal.org/}.
        Your task is to make the assertion succeed in the corresponding file.
        You may only change the definition of the solution, i.e., you should fill in the dots in the form of \verb|(def solution ...)| or \verb|(defn solution [...] ...)|.
    \end{exercise}
	
	
    \enlargethispage{2\baselineskip}
	\section*{Questions}
	If you have any questions, please contact Philipp K"orner (\texttt{p.koerner@hhu.de}) or post it to the Rocket.Chat group.
\end{document}

