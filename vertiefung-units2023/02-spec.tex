\documentclass[11pt,a4paper]{article}
\usepackage[ngerman]{babel}
\usepackage[TS1,T1]{fontenc}
\usepackage[utf8x]{inputenc}
\selectlanguage{ngerman}
\usepackage{theorem}
\usepackage[scaled=0.9]{helvet}
\usepackage{amsmath}
\usepackage{amssymb}
\usepackage{hyperref}
\usepackage{stmaryrd}
\usepackage{pgf,tikz}
\usepackage{relsize}
\usepackage{enumitem}
\usepackage{graphicx}
\usepackage{algpseudocode,amsmath,xifthen}

\newcounter{numb}
\theoremstyle{break}
\theorembodyfont{\upshape}
   	\newtheorem{exercise}{Exercise}[numb]
\setlength{\oddsidemargin}{0cm}
\setlength{\textwidth}{16cm}
\setlength{\textheight}{23cm}
\setlength{\topmargin}{-2cm}

\usetikzlibrary{shapes,arrows,automata,positioning,decorations.fractals}
\renewcommand\familydefault{\sfdefault}

\newcommand{\header}[2]{
\begin{minipage}[b]{\textwidth}
		\parbox[t]{5cm}{%
			\includegraphics[width=4cm]{unilogo}
			\hfill
		}
		\parbox[b]{11cm}{%
			%\scshape%
			Heinrich-Heine-University D\"usseldorf\\
			Computer Science Department\\
			Software Engineering and Programming Languages\\
			%Professor Dr.\ M.\ Leuschel
		Philipp K\"orner
}
		
\end{minipage}
	\begin{center}
		\bf
		Functional Programming -- WT 2024 / 2025\\
        Reading Guide {\thenumb}: {#1}
	\end{center}

    \pagestyle{empty}
    \paragraph{Timeline:} This unit should be completed by {#2}.
}

\setcounter{numb}{2}

\usepackage{upquote}

\begin{document}

\begin{minipage}[b]{\textwidth}
	\parbox[t]{5cm}{%
		\includegraphics[width=4cm]{unilogo}
		\hfill
	}
	\parbox[b]{11cm}{%
		%\scshape%
		Heinrich-Heine-University D\"usseldorf\\
		Computer Science Department\\
		Software Engineering and Programming Languages\\
		%Professor Dr.\ M.\ Leuschel
		Philipp K\"orner
	}
\end{minipage}
\begin{center}
	\bf
	Functional Programming -- ST 2023\\
	Reading Guide 02: clojure.spec
\end{center}

\pagestyle{empty}

\paragraph{Timeline:} This unit should be completed by 21.04.2023.

\section{Material} 

\begin{itemize}
    \item Talks (probably start here):
    \begin{itemize}
	\item Rich Hickey: clojure.spec \url{https://vimeo.com/195711510} --- good overview, bad audio quality
	\item Stuart Halloway: Agility \& Robustness: clojure.spec \url{https://www.youtube.com/watch?v=VNTQ-M_uSo8} --- good overview, alternative to Rich's talk that is easier on your ears
    \item Rich Hickey: Spec-ulation \url{https://www.youtube.com/watch?v=oyLBGkS5ICk} --- less on topic, more on the context of spec (optional)
    \end{itemize}

    \item Written Material:
    \begin{itemize}
	\item Material/repl2022-vertiefung/src/repl/14\_spec.clj --- overview and code examples
	\item Clojure Guide: spec \url{https://clojure.org/guides/spec} --- official reference, alternative to the REPL session
    \item Getting Clojure, chapter 15 --- modern book covering the material (alternative written material)
    \end{itemize}

\end{itemize}


\section{Learning Outcomes}

After completing this unit you should be able to

\begin{itemize}
    \item discuss advantages and disadvantages of the clojure.spec library.
    \item read specs and give example values that meet the given spec.
    \item write specs yourself.
\end{itemize}

\section{Highlights}

\begin{itemize}
    \item spec: \verb|def|, \verb|fdef|, \verb|valid?|,
    \item spec regex: \verb|cat|, \verb|alt|, \verb|+|, \verb|*|, \verb|?|, \verb|&|, \verb|spec|
    \item combining specs: \verb|and|, \verb|or|, \verb|map-of|, \verb|coll-of|, \verb|keys|
    \item conformed/unformed values
\end{itemize}



\section{Exercises}

\begin{exercise}[Specs]
\begin{enumerate}[label=\alph*)]
\item
Specify a spec for a player in a card game,
who should look something like this:

\begin{verbatim}
{:cards/name "Philipp"
 :cards/hand [[3 :clubs] [:ace :spades]]} 
\end{verbatim}

Any arbitrary string is allowed as a name.
A non-empty sequence of 2-tuples is associated under \verb|:cards/hand|.
The first entry is the value of the card, the second is the suit.

\item
Write a function \verb|discard|,
which receives a set of players
as well as a name and a playing card.
If this playing card is currently in the hand
of the named player, it is discarded.

\item
Annotate \verb|discard| with specs, which ensure
that if called correctly, a set of players
is returned by the function.

\item
Is it possible to detect a problem with generated tests?

Note: you can limit the size of the generated values with\\
\verb|(stest/check `discard {:clojure.spec.test.check/opts {:max-elements 4 :max-size 3}})|
Otherwise, testing can take a very long time..
\end{enumerate}
\end{exercise}

\begin{exercise}[Specs II]
    In exercise 9.2, a test.check generator was used to generate users with identifiers.
    Write a spec for a user.

    A user is represented by a map
    containing their first- and last name and an identifier.
    The identifier consists of the first two characters of the first name,
    the first three characters of the last name, and three digits.
    An example of a valid user is 
    
    \verb|{:first-name "John", :last-name "Wayne", :id "joway142"}|.

    Some invalid values are:
    \begin{itemize}
        \item \verb|42|
        \item \verb|{:first-name "John", :last-name "Wayne", :id "cowboystud69"} ; incorrect identifier|
        \item \verb|{:first-name "John", :surname "Wayne", :id "joway142"} ; :surname is invalid|
    \end{itemize}
\end{exercise}


\section*{Questions}
If you have any questions, please contact Philipp K"orner (\texttt{p.koerner@hhu.de}).
\end{document}

