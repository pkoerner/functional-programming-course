\documentclass[11pt,a4paper]{article}
\usepackage[ngerman]{babel}
\usepackage[TS1,T1]{fontenc}
\usepackage[utf8x]{inputenc}
\selectlanguage{ngerman}
\usepackage{theorem}
\usepackage[scaled=0.9]{helvet}
\usepackage{amsmath}
\usepackage{amssymb}
\usepackage{hyperref}
\usepackage{stmaryrd}
\usepackage{pgf,tikz}
\usepackage{relsize}
\usepackage{enumitem}
\usepackage{graphicx}
\usepackage{algpseudocode,amsmath,xifthen}

\newcounter{numb}
\theoremstyle{break}
\theorembodyfont{\upshape}
   	\newtheorem{exercise}{Exercise}[numb]
\setlength{\oddsidemargin}{0cm}
\setlength{\textwidth}{16cm}
\setlength{\textheight}{23cm}
\setlength{\topmargin}{-2cm}

\usetikzlibrary{shapes,arrows,automata,positioning,decorations.fractals}
\renewcommand\familydefault{\sfdefault}

\newcommand{\header}[2]{
\begin{minipage}[b]{\textwidth}
		\parbox[t]{5cm}{%
			\includegraphics[width=4cm]{unilogo}
			\hfill
		}
		\parbox[b]{11cm}{%
			%\scshape%
			Heinrich-Heine-University D\"usseldorf\\
			Computer Science Department\\
			Software Engineering and Programming Languages\\
			%Professor Dr.\ M.\ Leuschel
		Philipp K\"orner
}
		
\end{minipage}
	\begin{center}
		\bf
		Functional Programming -- WT 2024 / 2025\\
        Reading Guide {\thenumb}: {#1}
	\end{center}

    \pagestyle{empty}
    \paragraph{Timeline:} This unit should be completed by {#2}.
}

\setcounter{numb}{5}
\usepackage{csquotes}

\begin{document}

\begin{minipage}[b]{\textwidth}
	\parbox[t]{5cm}{%
		\includegraphics[width=4cm]{unilogo}
		\hfill
	}
	\parbox[b]{11cm}{%
		%\scshape%
		Heinrich-Heine-University D\"usseldorf\\
		Computer Science Department\\
		Software Engineering and Programming Languages\\
		%Professor Dr.\ M.\ Leuschel
		Philipp K\"orner \\
        Jens Bendisposto
	}
\end{minipage}
\begin{center}
	\bf
	Functional Programming -- ST 2023\\
    Reading Guide 05: Databases
\end{center}

\pagestyle{empty}

\paragraph{Timeline:} This unit should be completed by 12.05.2023.

\section{Material} 

\renewcommand{\labelenumi}{\Alph{enumi}}
\begin{itemize}    
	
	\item Rick Hickey: Datomic \url{https://www.youtube.com/watch?v=9TYfcyvSpEQ}
    (until 21:50: how can a data base be embedded into the epochal time model?)
	
	\item Rich Hickey: Writing Datomic in Clojure \url{https://www.youtube.com/watch?v=7Fi-UvrRpyI}
    (architecture of Datomic, in particular compared to other database systems)
	
    \item Datomic Datalog: \url{https://www.youtube.com/watch?v=bAilFQdaiHk}
    (brief introduction to the query language)
\end{itemize} 

Useful resources:

\begin{itemize}
	
	\item Interactive Datomic Tutorial: \url{https://max-datom.com/}
	
    \item Datahike (open-source alternative to Datomic for smaller projects): \url{https://github.com/replikativ/datahike}
\end{itemize}


\section{Learning Outcomes}

After completing this unit you should be able to

\begin{itemize}
    \item compare the architecture of Datomic to \enquote{traditional} data bases.
    \item describe unique features of a data base that conforms to the epochal time model.
    \item retrieve data using Datalog-style queries.
\end{itemize}



\section{Exercises}

\begin{exercise}[myMDB]

    For this task, you may use the \verb|05-skeleton.clj| you will find in ILIAS.
    Additionally, we use two data sets IMDb provides at \url{https://datasets.imdbws.com/}.
    We'll need to download and extract \verb|title.basics.tsv.gz| as well as \verb|title.ratings.tsv.gz|.

    \begin{enumerate}
        \item Feed the data into a Datehike data base.
            \verb|05-skeleton.clj| already contains code
            that parses the .tsv-file,
            sets up a data base and provides a schema.
            Your task here is to transform the data into a transaction.

        \item Query the data base and retrieve the following information:
            \begin{itemize}
                \item In what year was \enquote{Beverly Hills Cop} released?
                \item What is the IMDb score for \enquote{The Matrix}?
                \item Which movies from the 80s has a score of at least 9.0?
                \item What is the average score of all movies released in 1914?
            \end{itemize}
    \end{enumerate}


\end{exercise}


\end{document}

